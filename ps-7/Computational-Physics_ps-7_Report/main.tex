\documentclass[11pt]{article}
\usepackage{graphicx}
\usepackage{float}
\usepackage{hyperref}
\usepackage{natbib}
\usepackage{listings}
\usepackage{xcolor}
\usepackage[dvipsnames]{xcolor}
\usepackage[svgnames]{xcolor}
\usepackage{amsmath} % For the equation* environment
\usepackage{amssymb}

\hypersetup{
    colorlinks=true,
    linkcolor=red,
    filecolor=cyan,      
    urlcolor=orange,
    pdftitle={Overleaf Example},
    pdfpagemode=FullScreen,
    }

\setlength{\textwidth}{6.5in}
\setlength{\headheight}{0in}
\setlength{\textheight}{8.0in}
\setlength{\hoffset}{0in}
\setlength{\voffset}{0in}
\setlength{\oddsidemargin}{0in}
\setlength{\evensidemargin}{0in}

\lstdefinestyle{txtstyle}{
    basicstyle=\ttfamily\small,
    breaklines=true,
    backgroundcolor=\color{Bisque}
}
\lstset{style = txtstyle}

\definecolor{codegreen}{rgb}{0,0.6,0}
\definecolor{codegray}{rgb}{0.5,0.5,0.5}
\definecolor{codepurple}{rgb}{0.58,0,0.82}
\definecolor{backcolour}{rgb}{0.95,0.95,0.92}

\lstdefinestyle{mystyle}{
    backgroundcolor=\color{backcolour},   
    commentstyle=\color{codegreen},
    keywordstyle=\color{magenta},
    numberstyle=\tiny\color{codegray},
    stringstyle=\color{codepurple},
    basicstyle=\ttfamily\footnotesize,
    breakatwhitespace=false,         
    breaklines=true,                 
    captionpos=b,                    
    keepspaces=true,                                   
    numbersep=5pt,                  
    showspaces=false,                
    showstringspaces=false,
    showtabs=false,                  
    tabsize=2
}

\title{Computational Physics ps-7 Report}
  
\author{Tongzhou Wang, \\ GitHub account: TZW56203, repository: phys-ga2000. \\ \url{https://github.com/TZW56203/phys-ga2000}}

\date{October 27, 2024}

\begin{document}

\maketitle

\section{Problem 1}

\subsection{Part (a)}
By the Newton's second law we have
\begin{equation}
    \frac{GMm_s}{r^2} - \frac{Gmm_s}{(R-r)^2} = m_s \omega^2 r.
\end{equation}
Simplifying, we get the required expression
\begin{equation}\label{1a}
    \frac{GM}{r^2} - \frac{Gm}{(R-r)^2} = \omega^2 r.
\end{equation}

Since the satellite have the same angular velocity as the moon, we also have
\begin{equation}
    \frac{GMm}{R^2} = m \omega^2 R,
\end{equation}
which gives
\begin{equation}
    \omega^2 = \frac{GM}{R^3}.
\end{equation}

Plugging this in to (\ref{1a}), we get
\begin{equation}
    \frac{GM}{r^2} - \frac{Gm}{(R-r)^2} = \frac{GMr}{R^3}.
\end{equation}
Simplifying and taking $m^{\prime} = m/M$ and $r^{\prime} = r/R$, we get
\begin{equation}
    1 - r^{\prime 3} - m^{\prime} \left( \frac{r^{\prime}}{1 - r^{\prime}} \right) ^2 = 0.
\end{equation}

\subsection{Part (b)}
Listing \ref{lst:Lagrange} shows the Lagrange points in the following cases.
\lstinputlisting[caption={Lagrange points.}, label={lst:Lagrange}]{code/ps-7-1.txt}

\section{Problem 2}
Listing \ref{lst:min} shows the computed point at which the function $y = (x-0.3)^2 \exp(x)$ takes minimum. The self-written Brent's method gives a result close to that of \texttt{scipy.optimize.brent} and the true value 0.3.
\lstinputlisting[caption={Minimization.}, label={lst:min}]{code/ps-7-2.txt}

\end{document}
